\documentclass[preprint]{sigplanconf}

\usepackage{amssymb}
\usepackage{amsmath}
\usepackage{amsthm}
\usepackage{stmaryrd}
\usepackage{xcolor}
\usepackage{listings}
\usepackage{lstcoq}
\usepackage{comment}


\lstset{ %
  numberbychapter=false, %
  language=coq, %
%%  frame=lines, %
  frameshape={yyy}{n}{n}{yyy}, %
  framexleftmargin=-3pt,
  framexrightmargin=-3pt,
  numberstyle=\tiny, %
  basicstyle=\footnotesize, %
  captionpos=b,
  numbersep=5pt,
  xleftmargin=5pt,
  xrightmargin=5pt}

\usepackage[bookmarks=true,colorlinks=true, citecolor=cyan]{hyperref} %%linkcolor=MidnightBlue, 

\newcommand{\FOR}{{\tt for} \ }
\newcommand{\FORALL}{{\tt forall} \ }
\newcommand{\EXISTS}{{\tt exists} \ }
\newcommand{\WHERE}{{\tt where} \ }
\newcommand{\IN}{ \ {\tt in} \ }
\newcommand{\RETURN}{{\tt return} \ }
\newcommand{\DO}{{\tt do}}
\newcommand{\IF}{{\tt if} \ }
\newcommand{\THEN}{{\tt then} \ }
\newcommand{\ELSE}{{\tt else} \ }
\newcommand{\ZERO}{{\tt zero}}
\newcommand{\FALSE}{{\tt false}}
\newcommand{\BIND}{{\tt bind}}
\newcommand{\UNION}{{\tt union}}
\newcommand{\MAP}{{\tt map}}
\newcommand{\CONS}{{\tt Cons}}
\newcommand{\NIL}{{\tt Nil}}

\newcommand{\greg}[1]{\textcolor{blue}{GREG: #1}}
\newcommand{\ltac}[0]{\ensuremath{\mathcal{L}_{\mathrm{tac}}}}

\begin{document}

\title{Using Dependent Types and Tactics to Enable Semantic Optimization of Language-Integrated Queries}

\authorinfo{Gregory Malecha}{University of California at San Diego}{\sf gmalecha@eng.ucsd.edu} 

\authorinfo{Ryan Wisnesky\titlenote{Work supported by ONR grant N000141310260 and AFOSR grant FA9550-14-1-0031}}{Massachusetts Institute of Technology}{\sf wisnesky@math.mit.edu}

\date{\today}

\maketitle
%\vspace*{-.3in}
\begin{abstract}
Semantic optimization -- the use of data integrity constraints to optimize relational queries -- has been well studied but, owing to limitations in how SQL handles constraints, has not often been applied by mainstream RDBMSs. In a language-integrated query setting, however, the query provider is free to rewrite queries before they are executed on an RDBMS.  We show, using Coq as our ambient language, how to use dependent types to represent a well known class of constraints -- embedded, implicational dependencies -- and how Coq tactics can be used to implement a particular kind of semantic optimization: tableaux minimization, which minimizes the number of joins required by a query.
\end{abstract}

\section{Introduction}

{\it Semantic optimization}~\cite{foundations,Deutsch:2006:QRC:1121995.1122010,Popa99anequational} is the 
use of data integrity constraints such as keys, functional dependencies, inclusions, and join decompositions to optimize relational queries. For example~\cite{foundations}, consider the following contrived query over a relation (set of records) $Movies$ 
with fields ${\sf title}$, ${\sf director}$, and ${\sf actor}$:
\begin{eqnarray*}
& & \FOR (m_1 \IN Movies) \ (m_2 \IN Movies) \\
 & & \WHERE m_1.{\sf title} = m_2.{\sf title} \\
 & & \RETURN (m_1.{\sf director}, m_2.{\sf actor})
\end{eqnarray*}        
This query returns (a set of) tuples $(d,a)$ where $a$ acted in a movie directed by $d
$.  A naive implementation of this query will require a join.  However, when $Movies$ 
satisfies the the functional dependency ${\sf title} \to {\sf director}$ (meaning that 
if $({\sf director}: d, {\sf title}: t, {\sf actor}: a)$ and $({\sf director}: d^\prime, {\sf title}: t^\prime, {\sf actor}: a^\prime)$ are $Movies$ records such that $t = t^\prime$, then $d = d^\prime$), this query is equivalent to:
\begin{eqnarray*}
& & \FOR (m \IN Movies) \\
 & & \RETURN (m.{\sf director}, m.{\sf actor})
 \end{eqnarray*}
which can be evaluated without a join.  (Note that if $Movies$ did not satisfy the functional dependency, the equivalence would not necessarily hold.)  

Of course, knowing that the functional dependency holds, a programmer might simply write the optimized query to begin with.  But constraints are not always known at compile time, such as when relations are indexed dynamically.  Moreover, people are not always the programmers: information-integration systems such as Clio~\cite{haas:clio} automatically generate large numbers of queries.  The significant, potentially order-of-magnitude speed-ups enabled by semantic optimization are well-documented in the literature~\cite{Cheng:1999:ITS:645925.671357}.  

Although certain RDBMS's such as DB2 can perform limited amounts of semantic optimization~\cite{Cheng:1999:ITS:645925.671357}, RDBMS's are fundamentally limited by the expressiveness of SQL as a constraint specification language: SQL includes keys and foreign keys but constraints such as the functional dependency above are not directly expressible in SQL.  (Technically, functional dependencies can be encoded as {\tt CHECK} constraints, but even {\tt CHECK} constraints cannot capture multi-table constraints such as join decompositions).  In relational database theory, a fragment of first-order logic, the so-called {\it embedded, implicational dependencies} (EDs), are used to capture almost all constraints used in practice, including keys, foreign keys, inclusions, functional dependencies, and join decompositions, and a large body of literature has developed to facilitate reasoning about queries in the presence of EDs~\cite{Popa99anequational}. 

{\bf Contributions and Outline.} In this paper we demonstrate that dependently-typed language-integrated query systems (LINQs~\cite{monad}) that compile to SQL can expose data integrity constraints, in the guise of EDs, as first-class objects to their users, and that they can apply sophisticated semantic optimization techniques before translating user queries into SQL.  In particular, we show, using Coq~\cite{coq:coq} as our ambient language, how to use dependent equality types to represent EDs, and how to use Coq tactics to implement a particular kind of semantic optimization: tableaux minimization, which minimizes the number of joins required by a query.  This paper is divided into two parts: the first part is a tutorial on tableaux minimization, and the second part is a Coq rendering of the first part. The Coq development is available at {\sf github.com/gmalecha/semantic-query}.

{\bf Related Work.} Most theoretical work on language-integrated query systems is done in a simply-typed setting~\cite{tannen:1992:NEQ:645500.655920}.  In practice, however, sophisticated type systems are often used to to facilitate the embedding of a query sublanguage into a general purpose programming language.  For example, labelled row types~\cite{mpj:jones1996a} can be used to embed DBMS records into a programming language, and the Opaleye library for Haskell uses the Arrow type-class to statically enforce the wellformedness of its SQL output~\cite{opaleye}.  Rarer still are dependently-typed embedded query languages: although Coq has been used to prove the correctness of certain database-related languages, data structures, and algorithms~\cite{DBLP:conf/popl/DelawarePGC15}~\cite{Malecha:2010:TVR:1706299.1706329}~\cite{coqdb}, none of this work is concerned with using Coq directly as an embedded query language as we are doing in this paper (i.e., these works use deep embeddings of query languages, whereas we use a shallow embedding).  

\section{Queries}
%For ease of exposition, in the first part of this paper we will assume we are working in a strongly-normalizing typed $\lambda$-calculus with first-class records, such as~\cite{mpj:jones1996a}.

In this paper we will focus on relational {\it conjunctive queries}~\cite{foundations}, and for the first part of this paper the specifics of our query language will not matter.   We will write $(l_1: e_1, \ldots, l_N: e_N)$ to indicate a record with unique labels $l_1, \ldots l_N$ formed from expressions $e_1, \ldots, e_N$, where an expression has the form $v.l$ for a variable $v$ and label $l$.  We will abbreviate (potentially 0-length) vectors of variables $x_1,...,x_N$ as $\overrightarrow{x}$.  We will write $P(\overrightarrow{x})$ to indicate a conjunction of equalities over expressions over variables $\overrightarrow{x}$.  Assumed base relations (often called {\it roots}) will be written in capital letters, such as $\overrightarrow{X}$.  A {\it tableau} has the form:
\begin{normalsize}
\begin{eqnarray*}
 & & \FOR \overrightarrow{(x \IN X)} \\
 & & \WHERE P(\overrightarrow{x})
\end{eqnarray*}
\end{normalsize}
The $\overrightarrow{(x \IN X)}$ are called {\it generators}.  A (conjunctive) {\it query} is a pair of a  tableau and a record (``return clause'') $R(\overrightarrow{x})$:
\begin{normalsize}
\begin{eqnarray*}
 & & \FOR \overrightarrow{(x \IN X)} \\
& & \WHERE  P(\overrightarrow{x}) \\ 
 & & \RETURN R(\overrightarrow{x})
\end{eqnarray*}
\end{normalsize}
{\bf Extensions.} We will only consider relational conjunctive queries in this paper, but many extensions to conjunctive queries have been studied in the literature~\cite{foundations}.  Two extensions are particularly important, because many results about semantic optimization, including tableaux minimization, hold for these extensions~\cite{Popa99anequational}:
\begin{itemize} 
\item  It is possible to allow generators to be dependent, thereby allowing, for example, nested relations~\cite{Popa99anequational}:
\begin{normalsize}
$$ \FOR (g \IN Groups) \ (person \IN g) \ \ldots $$
\end{normalsize}
\item It is possible to interpret queries in arbitrary {\it monads with zeroes and pluses}, for example, the list monad or the bag monad.  However, the  optimization procedure described in this paper is only sound for monads that are commutative and idempotent~\cite{Popa99anequational}:
$$
\FOR (x \IN X)(y \IN Y)  \cong \FOR (y \IN Y) (x \IN X) 
$$
$$
\FOR (x \IN X) \cong \FOR (x \IN X)(x' \IN X) 
$$
Such monads arise, for example, as power monads on topoi~\cite{BW}.  It is also possible to interpret queries in {\it monad algebras}~\cite{755736}.  For example, it is possible to write a query to find the largest element of a set: 
\begin{normalsize}
$$ \FOR (x \IN SomeSetOfInts) \ {\tt max} \ x $$
\end{normalsize}

\end{itemize}

\section{Embedded Dependencies}

An {\it embedded dependency (ED)}~\cite{foundations} is a pair of tableaux, where one tableau is universally quantified, and the other existentially:
\begin{normalsize}
\begin{eqnarray*}
C & := & \FORALL \overrightarrow{(x \IN X)} \\
 & & \WHERE P(\overrightarrow{x}) \\
 & & \EXISTS \overrightarrow{(y \IN Y)} \\
 & & \WHERE B(\overrightarrow{x}, \overrightarrow{y})
\end{eqnarray*}
\end{normalsize}
{\bf Example.} The functional dependency from our example from the introduction is written (the \EXISTS clause is empty):
\begin{normalsize}
\begin{eqnarray*}
& & \FORALL (x \IN Movies) \ (y \IN Movies) \\
& & \WHERE x.{\sf title} = y.{\sf title}, \\ 
& & \EXISTS \\
& & \WHERE x.{\sf director} = y.{\sf director}
\end{eqnarray*}
\end{normalsize}
An ED $C$ gives rise to two conjunctive queries, the {\it front} and {\it back} of $C$.  We write $\mathcal{L}
(\overrightarrow{x})$ to indicate a record capturing the variables $\overrightarrow{x}
$; e.g., $({\sf x_1}: x_1, \ldots ,{\sf x_N}: x_N)$.  %The front of an ED is:
\begin{normalsize}
\begin{eqnarray*}
front(C)& := & \FOR \overrightarrow{(x \IN X)} \\ 
& & \WHERE P(\overrightarrow{x}) \\
& & \RETURN \mathcal{L}(\overrightarrow{x})  \\
& & \\
%\end{eqnarray*}
%\end{normalsize}
%and the back is
%\begin{normalsize}
%\begin{eqnarray*}
back(C) & := & \FOR \overrightarrow{(x \IN X)} \ \overrightarrow{(y \IN Y)} \\ 
& & \WHERE P(\overrightarrow{x}) \wedge B(\overrightarrow{x}, \overrightarrow{y}) \\
& & \RETURN \mathcal{L}(\overrightarrow{x})
\end{eqnarray*}
\end{normalsize}
It is easy to establish~\cite{Popa99anequational} that for every $I$,
$$I \models C \ \ \ \ \textnormal{iff} \ \ \ \  front(C)(I) = back(C)(I)$$
In fact, in the second half of this paper, we will use a dependent equality type corresponding to the above equation as a type of proofs that an ED holds in a particular instance.

{\bf Notation.}  When two queries $Q_1$ and $Q_2$ give the same result on every instance, we write $Q_1 \cong Q_2$.  When $Q_1$ and $Q_2$ give the same result on every instance satisfying some set of EDs $C$, we write $C \vdash Q_1 \cong Q_2$.


%
%In the set monad, the above definition of satisfaction corresponds to our intuitive notion of satisfaction; however, this definition of satisfaction has the advantage of being definable for every monad with zero. 
%
%Continuing with our example, our functional dependency holds of a particular instance $Movies$ when
%\begin{normalsize}
%\begin{eqnarray*}
%& & \FOR (x \IN Movies) \ (y \IN Movies) \\
%& & \WHERE x.{\sf title} = y.{\sf title}, \\ 
%& & \RETURN ({\sf x}: x, {\sf y}: y) \\
%& = &  \\
%& & \FOR (x \IN Movies) \ (y \IN Movies) \\
%& & \WHERE x.{\sf title} = y.{\sf title} \wedge x.{\sf director} = y.{\sf director} \\ 
%& & \RETURN ({\sf x}: x, {\sf y}: y) 
%\end{eqnarray*}
%\end{normalsize}
%\noindent
%For example, in this instance:
%\begin{normalsize}
%\begin{eqnarray*}
%{\sf title} & {\sf director} & {\sf actor} \\
%T & D & A \\ 
%T & D & B
%\end{eqnarray*}
%\end{normalsize}
%%\vspace{-.4in}
%\noindent
%the constraint holds because both sides evaluate to (omitting some labels to 
%save space):
%\begin{normalsize}
%\begin{eqnarray*}
%{\sf x} & {\sf y} &  \\
%(T, D, A) & (T, D, A) & \\ 
%(T, D, A) & (T, D, B) & \\ 
%(T, D, B) & (T, D, A) & \\ 
%(T, D, B) & (T, D, B) & 
%\end{eqnarray*}
%\end{normalsize}
%\noindent
%whereas in this instance:
%\begin{normalsize}
%\begin{eqnarray*}
%{\sf title} & {\sf director} & {\sf actor} \\ 
%T & D_1 & A \\ 
%T & D_2 & B
%\end{eqnarray*}
%\end{normalsize}
%\noindent
%the constraint does not hold because the left-hand side and right-hand side evaluate to, respectively:
%\begin{normalsize}
%
%\parbox{3in}{
%\begin{eqnarray*}
%{\sf x} & {\sf y} &  \\
%(T, D_1, A) & (T, D_1, A) &    \\ 
%(T, D_1, A) & (T, D_2, B) &  \\ 
%(T, D_2, B) & (T, D_1, A) & \\ 
%(T, D_2, B) & (T, D_2, B) & 
%\end{eqnarray*}
%}
%\parbox{3in}{
%\begin{eqnarray*}
%{\sf x} & {\sf y} &  \\
% (T, D_1, A) & (T, D_1, A) & \\ 
% (T, D_2, B) & (T, D_2, B) & \\ 
%\end{eqnarray*}
%}
%
%\end{normalsize}
%

\section{Homomorphisms}

A {\it homomorphism} between queries, $h : Q_1 \to Q_2$ 
\begin{normalsize}
\begin{eqnarray*}
Q_1 & := & \FOR \overrightarrow{(v_1 \IN V_1)} \\
          & & \WHERE P_1(\overrightarrow{v_1}) \\
          & & \RETURN R_1(\overrightarrow{v_1}) \\
\to_h & & \\        
Q_2 & := & \FOR \overrightarrow{(v_2 \IN V_2)} \\
          & & \WHERE P_2(\overrightarrow{v_2}) \\
          & & \RETURN R_2(\overrightarrow{v_2})
\end{eqnarray*}
\end{normalsize}
is a substitution mapping the $\FOR$-bound variables of $Q_1$ (namely, $
\overrightarrow{v_1}$) to the $\FOR$-bound variables of $Q_2$ (namely, $
\overrightarrow{v_2}$) that preserves the structure of $Q_1$ in the sense that
\begin{itemize}
\item  
 $(h(v_{1_i}) \IN V_{1_i})$ $ \in$ $\overrightarrow{(v_2 \IN V_2)}$ (that is, the image of each generator in $Q_1$ is found in the generators of $Q_2$). 

\item $P_2(\overrightarrow{v_2})$ $\vdash$ $P_1(h(\overrightarrow{v_1}))$  (that is, the image of the where clause of $Q_1$ is entailed by the where clause of $Q_2$).

\item $P_2$ $\vdash$ $R_1(h(\overrightarrow{v_1})) = R_2(\overrightarrow{v_2})$ (that is, the image of the return clause of $Q_1$ is equal, under $P_2$, to the return clause of $Q_2$).
\end{itemize}
A homomorphism of tableaux is defined the same way, except that the condition about $\RETURN$ clauses is dropped.  

{\bf Notation.} We write $Q_1 \leftrightarrow Q_2$ to mean that there exists homomorphisms $Q_1 \to Q_2$ and $Q_2 \to Q_1$ and we say that $Q_1$ and $Q_2$ are {\it homomorphically equivalent}.  The existence of a homomorphism $Q_1 \to Q_2$ implies that for every $I$, $Q_2(I) \subseteq Q_1(I)$, and vice versa~\cite{foundations}.  Hence $Q_1 \cong Q_2$ iff $Q_1 \leftrightarrow Q_2$.

{\bf Example.} Consider our $Movies$ query 
%When queries are {\it path-conjunctive}---that is, when $P_1$, $P_2$ are conjunctions of equalities between paths of the form $v.l_1,.\ldots l_n, $ and  $R_1$ and $R_2$ are records built from paths, as we are assuming in this paper, finding homomorphisms is NP-hard.  Moreover, in this case there are practical, sound heuristics~\cite{Deutsch:2006:QRC:1121995.1122010} based on pruning the search space of substitutions to remove candidates that are ``obviously wrong'' based on a partial variable assignment. 
\begin{normalsize}
\begin{eqnarray*}
Q_1 & := & \FOR (m_1 \IN Movies) \ (m_2 \IN Movies) \\
 & & \WHERE m_1.{\sf title} = m_2.{\sf title} \\
 & & \RETURN (m_1.{\sf director}, m_2.{\sf actor})
\end{eqnarray*}   
\end{normalsize}
and the semantically optimized query:
\begin{normalsize}
\begin{eqnarray*}
Q_2 & := & \FOR (m \IN Movies) \\
 & & \RETURN (m.{\sf director}, m.{\sf actor})
\end{eqnarray*}   
\end{normalsize}
It is easy to see that for every $I$, $Q_2(I) \subseteq Q_1(I)$, and indeed there is a homomorphism $h : Q_1 \to Q_2$; namely, the substitution $m_1 \mapsto m, m_2 \mapsto m$.  To check this, we first apply $h$ to $Q_1$:
\begin{normalsize}
\begin{eqnarray*}
h(Q_1) & := & \FOR (m \IN Movies) \ (m \IN Movies) \\
 & & \WHERE m.{\sf title} = m.{\sf title} \\
 & & \RETURN (m.{\sf director}, m.{\sf actor})
\end{eqnarray*}   
\end{normalsize}
In $h(Q_1)$ each generator $(m \IN Movies)$ appears in $Q_2$.  Moreover, the ${\tt where}$ 
clause of $h(Q_1)$ is a tautology and hence is entailed by the (empty) ${\tt where}$ clause of 
$Q_2$.  Finally, the two ${\tt return}$ clauses are equal.  As such, the substitution $m_1 
\mapsto m, m_2 \mapsto m$ is a homomorphism.  

There is no homomorphism $Q_2 \to Q_1$, and hence $Q_1 \ncong Q_2$.  
There are only two candidates: $m \mapsto m_1$ and $m \mapsto m_2$.  
Neither works because neither of the  images of $Q_2$'s $\RETURN$ clause (neither 
$\RETURN (m_1.{\sf director}, m_1.{\sf actor})$ nor $\RETURN (m_2.{\sf director},$ $m_2.
{\sf actor})$) is equivalent to $Q_1$'s $\RETURN$ clause ($\RETURN (m_1.{\sf director},$ 
$m_2.{\sf actor})$), even under the equality in $Q_1$ ($m_1.{\sf title} = m_2.{\sf 
title}$). 

%  Indeed, consider the instance:  
%\begin{normalsize}
%\begin{eqnarray*}
%{\sf title} & {\sf director} & {\sf actor} \\ 
%T & D_1 & A \\ 
%T & D_2 & B
%\end{eqnarray*}
%\end{normalsize}
%$Q_1$ and $Q_2$ evaluate to, respectively
%
%\parbox{3in}{
%\begin{normalsize}
%\begin{eqnarray*}
% & {\sf director} & {\sf actor} \\ 
%& D_1 & A \\ 
%& D_1 & B \\
%& D_2 & A \\
%& D_2 & B
%\end{eqnarray*}
%\end{normalsize}
%} \ \ \ \ \parbox{3in}{
%\begin{normalsize}
%\begin{eqnarray*}
% & {\sf director} & {\sf actor} \\ 
%& D_1 & A \\ 
%& D_2 & B
%\end{eqnarray*}
%\end{normalsize}
%}
%
%Of course, if we had chosen an instance $I$ that satisfied the functional dependency {\sf Title} $\to$ {\sf Director}, then $Q_1(I)$ and $Q_2(I)$ would have evaluated to the same result.


\section{The Chase}
\label{sec:chase}

The chase is a confluent rewriting procedure that rewrites queries using EDs~\cite{foundations}.   Let

 %We now describe the chase, and in the next section we show how to use it to optimize queries.   
\parbox{1.5in}{
\begin{eqnarray*}
 C & := & \FORALL \overrightarrow{(x \IN X)} \\
 & & \WHERE P(\overrightarrow{x}) \\
 & & \EXISTS \overrightarrow{(y \IN Y)} \\
 & & \WHERE B(\overrightarrow{x}, \overrightarrow{y})
\end{eqnarray*}
} \ \ \ \ \ \parbox{1.5in}{
\begin{eqnarray*}
Q & := & \FOR \overrightarrow{(v \IN V)} \\
 & & \WHERE  O(\overrightarrow{v}) \\ 
 & & \RETURN R(\overrightarrow{v})
\end{eqnarray*}
}
and suppose there exists a (tableau) homomorphism $h : front(C) \to Q$.  A {\it chase step} is to rewrite $Q$ into $step(C,Q)$ by adding the image of the existential part of $C$:
\begin{normalsize}
\begin{eqnarray*}
step(C,Q) & := & \FOR \overrightarrow{(v \IN V)} \ \overrightarrow{(y \IN Y)} \\
 & & \WHERE  O(\overrightarrow{v}) \wedge B(\overrightarrow{h(x)}, \overrightarrow{y}) 
\\ 
 & & \RETURN R(\overrightarrow{v})
\end{eqnarray*}
\end{normalsize}
Chase steps are semantics-preserving on instances that obey the constraints~\cite{Popa99anequational}:
$$C \vdash Q \cong step(C,Q)$$
The {\it chase} itself is to repeatedly (and non-deterministically) choose a homomorphism and step until a fixed point (up to homomorphic equivalence) is reached:
$$
Q \rightsquigarrow step(C, Q) \rightsquigarrow step(C, step(C, Q)) \rightsquigarrow \ldots 
$$
%The termination condition is to not take a chase step when there is a homomorphism extending $h$ from $chase(Q, C)$ to $Q$.
Termination of the chase is undecidable, but if it terminates the final result is unique (up to homomorphic equivalence)~\cite{Deutsch:2006:QRC:1121995.1122010}.  Provided certain fairness conditions are met~\cite{Deutsch:2006:QRC:1121995.1122010}, the chase extends easily to sets of EDs by choosing a particular ED to chase with at each step.  A key theorem about the chase is that it reduces the question of query equivalence under constraints to simple homomorphic equivalence: let $C$ be a set of EDs and $Q_1, Q_2$ queries.  Then:
$$
C \vdash Q_1 \cong Q_2 \ \ \ \  \textnormal{iff} \ \ \ \ chase(C,Q_1) \leftrightarrow chase(C, Q_2)
$$
{\bf Example.} Continuing with our $Movies$ example, there is a homomorphism $x \mapsto 
m_1, y \mapsto m_2$ from the front of our constraint: 
\begin{normalsize}
\begin{eqnarray*}
C& := & \FORALL (x \IN Movies) \ (y \IN Movies) \\
& & \WHERE x.{\sf title} = y.{\sf title}, \\ 
& & \EXISTS \\
& & \WHERE x.{\sf director} = y.{\sf director}
\end{eqnarray*}
\end{normalsize}
to our original query:
\begin{normalsize}
\begin{eqnarray*}
Q_1 & := & \FOR (m_1 \IN Movies) \ (m_2 \IN Movies) \\
 & & \WHERE m_1.{\sf title} = m_2.{\sf title} \\
 & & \RETURN (m_1.{\sf director}, m_2.{\sf actor})
\end{eqnarray*}
\end{normalsize}
Hence, we can take a chase step: 
\begin{normalsize}
\begin{eqnarray*}
step(C, Q_1) := & & \FOR (m_1 \IN Movies) \ (m_2 \IN Movies) \\
 & & \WHERE m_1.{\sf title} = m_2.{\sf title} \wedge \\
 & & \ \ \ \ \ \ \ \ \ \ \ \  m_1.{\sf director} = m_2.{\sf 
director} \\
 & & \RETURN (m_1.{\sf director}, m_2.{\sf actor})
\end{eqnarray*}
\end{normalsize}
At this point we stop chasing, because $step(C, step(C, Q_1))$ is syntactically equal (and hence homomorphically equivalent) to $step(C, Q_1)$, and we have established that $C \vdash Q_1 \cong chase(C, Q_1)$. %In general, it is not enough to check for the syntactic equality of $chase(Q, C)$ and $Q$ to stop the chase, as queries can be equivalent without being syntactically equal.  Hence, we must use homomorphisms to detect termination.  

\section{Tableaux Minimization}
\label{sec:minimize}

We now demonstrate how to minimize queries in the presence of 
EDs, a technique known as ``tableaux minimization using chase and back-chase''~\cite{Deutsch:2006:QRC:1121995.1122010}.   Suppose we are given a query $Q$ and set of EDs $C$.  
We first chase $Q$ with $C$ to obtain $U$, a so-called {\it universal plan}.  We then 
search for subqueries of $U$ (which are intuitively obtained by removing generators from $U$), chasing each in turn with $C$ to check for equivalence with $U$.  There will always be a unique minimal query (up to homomorphic equivalence)~\cite{Deutsch:2006:QRC:1121995.1122010}.

\subsection*{Example - Movies}
Start with our query and constraint from the introduction:
\begin{normalsize}
\begin{eqnarray*}
Q_1 & := & \FOR (m_1 \IN Movies) \ (m_2 \IN Movies) \\
 & & \WHERE m_1.{\sf title} = m_2.{\sf title} \\
 & & \RETURN (m_1.{\sf director}, m_2.{\sf actor})
\end{eqnarray*}        
\begin{eqnarray*}
C & := & \FORALL (x \IN Movies) \ (y \IN Movies) \\
& & \WHERE x.{\sf title} = y.{\sf title} \\ 
& & \EXISTS \\
& & \WHERE x.{\sf director} = y.{\sf director}
\end{eqnarray*}
\end{normalsize}
The universal plan, i.e., $chase(C,Q_1)$, is:
\begin{normalsize}
\begin{eqnarray*}
U & := & \FOR (m_1 \IN Movies) \ (m_2 \IN Movies) \\
 & & \WHERE m_1.{\sf title} = m_2.{\sf title} \wedge m_1.{\sf director} = m_2.{\sf 
director} \\
 & & \RETURN (m_1.{\sf director}, m_2.{\sf actor})
\end{eqnarray*}        
\end{normalsize}
We proceed with tableau minimization by searching for subqueries of $U$.  
Removing the generator $(m_1 \IN Movies)$ and replacing $m_1$ with $m_2$ in the body of $Q$ gives a smaller query:
\begin{normalsize}
\begin{eqnarray*}
Q_2 & := & \FOR (m_2 \IN Movies) \\
 & & \RETURN (m_2.{\sf director}, m_2.{\sf actor})
\end{eqnarray*}        
\end{normalsize}
We wish to check that $C \vdash Q_1 \cong Q_2$, so we check that $U = chase(C,Q_1) \leftrightarrow chase(C, Q_2)$.  We find that $chase(C, Q_2) \cong Q_2$, so we will actually check that  $U \leftrightarrow Q_2$.  The identity substitution is a homomorphism $Q_2 \to U$: the important part to notice is the $\RETURN$ clause, wherein $(m_2.{\sf director},$ $m_2.{\sf actor})$ is equal to $(m_1.{\sf director},$ $m_2.{\sf actor})$ precisely because 
of the equality $m_1.{\sf director}$ $=$ $m_2.{\sf director}$, which appears in $U$ 
but not in $Q_1$.  There is also a homomorphism $U \to Q_2$, namely, $m_2 
\mapsto m, m_1 \mapsto m$.  We thus conclude that $C \vdash U \cong Q_2 \cong Q_1$. 
\newpage
\subsection*{Example - Indexing}

As we remarked in the introduction, a reasonably competent programmer might be able to optimize our $Movies$ query directly, without applying the chase at all.  But sometimes constraints are not available to the programmer, such as when indices are generated dynamically.  Consider the following query, which returns the names of all $People$ between 16 and 18 years old:
\begin{normalsize}
\begin{eqnarray*}
Q_1 & := & \FOR (p \IN People) \\
 & & \WHERE p.{\sf age} > 16 \wedge p.{\sf age} < 18 \\
 & & \RETURN p.{\sf name}
\end{eqnarray*}   
\end{normalsize}
Technically, this query is not a purely conjunctive query because the where clause involves the less-than predicate $<$.  However, the machinery of tableaux minimization can still be used, and one of the advantages of our Coq development is that users are free to write arbitrary Coq expressions in where clauses, and Coq tactics can be used to reason about such where clauses.
  
Depending on the underlying access patterns, or the whims of a database administrator, an RDBMS might transparently index $People$ by creating another relation $Children$, such that the following two constraints hold:
\begin{normalsize}
\begin{eqnarray*}
C_1 & := & \FORALL (p \IN People) \\
 & & \WHERE p.{\sf age} < 21 \\
 & & \EXISTS (c \IN Children) \\
 & & \WHERE p.{\sf name} = c.{\sf name} \wedge p.{\sf age} = c.{\sf age}  \\
 & & \\
  C_2 & := & \FORALL (c \IN Children) \\
  & & \WHERE \\
 & & \EXISTS (p \IN Person) \\
 & & \WHERE p.{\sf name} = c.{\sf name} \wedge p.{\sf age} = c.{\sf age}
\end{eqnarray*}       
\end{normalsize}

In order to use this new index queries written against $People$ must be rewritten, at runtime, to use $Children$.  Tableaux minimization provides an automated mechanism to do so. 

Let $C = \{ C_1, C_2\}$.  First, we find the universal plan $U = chase(C, Q_1)$.  We begin by chase stepping $Q$ with $C_1$.  The identity substitution is a homomorphism $front(C_1) \to Q_1$, because $p.{\sf age} < 21$ is entailed by $p.{\sf age} > 16 \wedge p.{\sf age} < 18$; thus we chase step to:
\begin{normalsize}
\begin{eqnarray*}
U & := & \FOR (p \IN People) \ (c \IN Children) \\
 & & \WHERE p.{\sf age} > 16 \wedge p.{\sf age} < 18 \wedge \\
 & & \ \ \ \ \ \ \ \ \ \ \ \ p.{\sf name} = c.{\sf name} \wedge p.{\sf age} = c.{\sf age} \\
 & & \RETURN p.{\sf name}
\end{eqnarray*}  
\end{normalsize}
and we find that $U \cong step(C_1, U)$, so no further chase steps using $C_1$ are possible.  Now we chase step $U$ using $C_2$, and we find that $U \cong step(C_2, U)$, so no further chase steps with $C_2$ are possible.  Hence we have computed the universal plan $U = chase(C,Q_1)$.

Next, we minimize the universal plan by removing the $(p \IN People)$ generator (note that to do so we must replace each occurrence of $p$ with some other well-typed variable, in this case $c$):
\begin{normalsize}
\begin{eqnarray*}
Q_2 & := & \FOR (c \IN Children) \\
 & & \WHERE c.{\sf age} > 16 \wedge c.{\sf age} < 18 \\
  & & \RETURN c.{\sf name}
\end{eqnarray*}  
\end{normalsize}
We now ``back-chase'' $Q_2$ with $C$.  We can take no chase steps with $C_1$, because there is no substitution $h$ that makes $(h(p) \IN People)$ equal to $(c \IN Children)$.  We can chase step with $C_2$ using the identity substitution to obtain:
\begin{normalsize}
\begin{eqnarray*}
Q_2' & := & \FOR (c \IN Children) \ (p \IN Person) \\
 & & \WHERE c.{\sf age} > 16 \wedge c.{\sf age} < 18 \wedge \\
 & & \ \ \ \ \ \ \ \ \ \ \ \   p.{\sf name} = c.{\sf name} \wedge p.{\sf age} = c.{\sf age}\\
  & & \RETURN c.{\sf name}
\end{eqnarray*}  
\end{normalsize}
and no further steps with $C_1$ or $C_2$ are possible. Hence we have computed $Q_2' = chase(C, Q_2)$.  Recall that our goal is to check that $C \vdash Q_1 \cong Q_2$, which we do by checking $U = chase(C, Q_1) \leftrightarrow chase(C, Q_2) = Q_2'$; i.e., by checking $U \leftrightarrow Q_2'$.  It's easy to see that $U$ and $Q_2'$ are homomorphically equivalent under the substitution $p \mapsto c, c \mapsto p$, and we are finished.   

%
%We check that $C \vdash Q^\prime \cong U$ by checking $chase(C, Q') \leftrightarrow U = chase(C,Q)$.  We start by finding $chase(C, Q')$, which turns out to be $Q'$ because no chase steps can be taken: there is no substitution $h$ that makes $(h(p) \IN People)$ equal to $(c \IN Children)$.  By the same reasoning there is no homomorphism $U \to Q^\prime$, and hence $C$ $\nvdash$ $Q^\prime \cong U$.  Indeed, there may be extra tuples in $Children$ that do not appear in $People$.  
%
%Fortunately, if our index was built correctly we know that an additional constraint holds:
%\begin{normalsize}
%\begin{eqnarray*}
%\end{eqnarray*}     
%\end{normalsize}
%As such, we may chase $Q^\prime$ with $C^\prime$ (using the identity substitution) to obtain the equivalent (under $C'$):
%\begin{normalsize}
%\begin{eqnarray*}
%Q^{\prime\prime} & := & \FOR (c \IN Children) \ (p \IN Person) \\
% & & \WHERE c.{\sf age} > 16 \wedge c.{\sf age} < 18 \wedge \\
% & & \ \ \ \ \ \ \ \ \ \ \ \   p.{\sf name} = c.{\sf name} \wedge p.{\sf age} = c.{\sf age}\\
%  & & \RETURN c.{\sf name}
%\end{eqnarray*}  
%\end{normalsize}
%Now we can see that the identity substitution is a homomorphism $Q^{\prime\prime} \leftrightarrow U$ (owing to the fact that $p.{\sf name} = c.{\sf name}$ and $p.{\sf age} = c.{\sf age}$), and since $chase(C',U) \cong U$, we know that $C, C' \vdash Q^{\prime\prime} \cong U$.  We established earlier that $C \vdash U \cong Q$ and that $C' \vdash Q'' \cong Q'$.  These facts allow us to conclude that $C, C' \vdash U \cong Q \cong Q' \cong Q''$.
%\vpsace{-.1in}
\section{Coq Development - Overview}

In the rest of this paper we demonstrate how to shallowly embed relational conjunctive queries into Coq and how to use dependent types and tactics to implement tableaux minimization as described in the first half of this paper.  We will continue to use the running movies example from the first half of the paper.  That query is expressed in Coq as:
\begin{coq}
Definition Movies : set (string * string * string) := ...

Definition title x := fst x.
Definition director x := fst (snd x).
Definition actor x := snd (snd x).

Definition q : set (string * string) :=
  m1 <- Movies ; m2 <- Movies ;
  guard (title m1 = title m2) ;
  return (director m1, actor m2).
\end{coq}
Here, \coqe{q} is the name of the query.  Our Coq query syntax is inspired by Haskell's syntax for monadic computations, and Coq's ``notation'' mechanism allows us to define infix operations such as \coqe{<-}.  Intuitively, \coqe{<-} means {\tt for}, \coqe{guard} means {\tt where}, and \coqe{return} means {\tt return}.  To the right of the colon is the query's type, \coqe{set (string * string)}, which represents a set that contains pairs of strings, and our Coq development is parametric in the implementation of the underlying type of sets.  For example, we can choose \coqe{set x := list x} and implement sets as lists, or choose \coqe{set x := x -> Prop} and implement sets as Coq ``ensembles''.  In fact, \coqe{set} need not even be a set monad; it need only be an arbitrary commutative, idempotent monad with zero and plus, which is why much of our Coq code refers to \coqe{M} (for monad) rather than \coqe{set}; see Figure~\ref{fig:chaseable-functor}.

%While this flexible syntax may make it easier to structure large queries, there are other benefits to it that we will see in Section~\ref{sec:low-level}.
Given the above query, and a representation of the functional dependency from the introduction, \coqe{title_director_ed}, which we elide for now, we can ask Coq to automatically construct the minimized query as follows:
\begin{coq}
Definition optimized_query: 
{q$_{opt}$ : M (string * string) | title_director_ed $\vdash$ q$_{opt}$ = q}.
optimize.
Defined.
\end{coq}
%Again the first line declares the name of a term.
The type of \coqe{optimized_query} says that \coqe{optimized_query} is a pair of a query, \coqe{q$_{opt}$}, and a proof that \coqe{q$_{opt}$} is equivalent to \coqe{q} on instances satisfying \coqe{title_director_ed}.  Here $\vdash$ is an infix operation that defines when two queries are equal modulo some constraint.  The actual Coq term corresponding to \coqe{optimized_query} is \coqe{Defined} by the \coqe{optimize} tactic.  We can see the result of the optimization by asking Coq to print the first component of the pair:
\begin{coq}
Eval compute in (proj1_sig optimized_query).
(* = x <- Movies ; return (director x, actor x)
 *   : set (string * string)   *)
\end{coq}
%and we see that the minimized query computed by Coq is indeed correct.

%Here, ``...'' elide the reduction strategy which tells Coq which symbols to keep abstract during reduction and the \coqe{proj1_sig} simply forgets the proof component.

 
%Now, the the right of the colon we have a more interesting type, noteably a dependent one.
%This syntax represents a dependent pair of a query (\coqe{q$_{opt}$}) of type \coqe{M (string * string)} and a proof of the proposition to the right of the vertical bar.
%Namely, this proof carries around a witness that under the assumption that the \coqe{title_director_ed} is provable about the database, the meaning of \coqe{q$_{opt}$} is the same as the meaning of \coqe{q}.
%Since we with Coq to fill in this value for us, we end the definition with a period and use the ``tactic'' \coqe{optimize} to discharge fill in the appropriate value.
%The keyword \coqe{Defined} signals the end of the definition since \coqe{optimize} completely fills in the term.


\paragraph{Querying in Coq.} All of the queries described in this paper are executable within Coq, using an implementation of sets as lists.  Even on small examples (5 rows), the optimized query above performs 2x faster than the non-optimized version.
While a production database would likely peform considerably faster, and on considerably larger relations, the semantic optimization that we achieve in this example enables us to reduce the query's asymptotic running time \emph{in a fully verified way}.

\paragraph{Query Normalization.} In the first half of the paper, all of our queries were normalized into single {\tt for} clause, a single {\tt where} clause, and a single {\tt return} clause.  But in a language-integrated query system, we can relax this requirement.  For example, we could introduce an additional guard condition after ``binding'' \coqe{m1}.
\begin{coq}
Definition q_LOR : set (string * string) :=
  m1 <- Movies ;
  guard (title m1 = ``Lord of the Rings'') ;
  m2 <- Movies ;
  guard (title m1  = title m2 ) ;
  return (director m1, actor m2).
\end{coq}
As is well-known~\cite{monad}, monadic computations such as relational conjunctive queries can always be normalized into the flat form used in the first half of this paper, and our Coq library does this normalization (in a fully verified way) during optimization.

\subsection{Queries and Constraints in Coq}

\begin{figure}[t]
\label{fig:chaseable-functor}
\begin{coq}
Class DataModel (M : Type -> Type) : Type :=
{ Mret  : forall {T}, T -> M T
; Mbind : forall {T U}, M T -> (T -> M U) -> M U (* Mbind m k = x <- m ; k *)
; Mzero : forall {T}, M T
; Mstr : forall {T U}, M T -> M U -> M (T * U) :=
     fun _ _ m1 m2 => Mbind m1 (fun x => Mbind m2 (fun y => Mret (x,y)))
; Mguard : forall {T}, bool -> M T -> M T :=
     fun _ P m => if p then m else Mzero
 (* plus many axioms *)    
}.
\end{coq}
%; Mimpl : forall {T}, M T -> M T -> Prop

%%   (** theorems **)
%% ; Reflexive_Mimpl : forall {A}, Reflexive (@Mimpl A)
%% ; Transitive_Mimpl : forall {A}, Transitive (@Mimpl A)

%% ; Proper_Mbind_impl : forall {A B},
%%     Proper (Mimpl ==> (pointwise_relation _ Mimpl) ==> Mimpl) (@Mbind A B)
%% ; Proper_Mret_impl : forall {A},
%%     Proper (eq ==> Mimpl) (@Mret A)

%% ; Mbind_assoc : forall {A B C} (c1 : M A) (c2 : A -> M B) (c3 : B -> M C),
%%     Meq (Mbind (Mbind c1 c2) c3)
%%         (Mbind c1 (fun x => Mbind (c2 x) c3))
%% ; Mbind_Mret : forall {A B} (x : A) (c : A -> M B),
%%     Meq (Mbind (Mret x) c) (c x)
%% ; Mret_Mbind : forall {A} (c : M A),
%%     Meq (Mbind c Mret) c
%% ; Mbind_Mzero : forall {A B : Type} (x : A -> M B), Meq (Mbind Mzero x) Mzero
%% ; Mbind_ignore : forall {T U} (x : M T) (y : M U),
%%               Mimpl (Mbind x (fun _ => y)) y
%% ; Mimpl_Mzero : forall {T} (c : M T), Mimpl Mzero c

%% ; Mbind_perm : forall {T U V} (m1 : M T) (m2 : M U) (f : T -> U -> M V),
%%     Meq (Mbind m1 (fun x => Mbind m2 (f x)))
%%         (Mbind m2 (fun y => Mbind m1 (fun x => f x y)))
%% ; Mbind_dup : forall {T U} (m : M T) (f : T * T -> M U),
%%     Mimpl (Mbind m (fun x => f (x,x)))
%%           (Mbind m (fun x => Mbind m (fun y => f (x,y))))

%% ; chaseable : forall (S S' T U : Type)
%%     (P : M S) (C : S -> bool) (E : S -> T)
%%     (F : M S') (Gf : S' -> bool) (B : M U) (Gb : S' -> U -> bool) :
%%     (edc : embedded_dependency F Gf B Gb)
%%     (h : S -> S'),
%%     Mimpl (Mmap h P) F ->
%%     forall x, C x = true -> Gf (h x) = true ->
%%     Meq (query P C E)
%%         (query (Mstr P B)
%%                (fun ab => C (fst ab) && Gb (h (fst ab)) (snd ab))
%%                (fun ab => E (fst ab))).

\caption{Collections Monads in Coq}
\label{fig:chaseable-functor}
\end{figure}

Our Coq library is parametric in an underlying type of sets, and hence we need an interface to interact with various set implementations.  Our interface is based monads~\cite{monad}, which are a useful interface for many kinds of collections, including sets~\cite{monad}.  The operations, but not all required axioms, of our monadic interface are shown in Figure~\ref{fig:chaseable-functor}.  Intuitively, in a set monad these operations are:
\begin{itemize}
\item \coqe{Mret v} is the function that injects \coqe{v} into the singleton set containing only \coqe{v}.
\item \coqe{Mbind m k} is the function that unions all sets \coqe{k x} for every \coqe{x} in the set \coqe{m}.  We will write \coqe{x <- m ; k} for \coqe{Mbind x (fun x => k)} where \coqe{x} occurs free in \coqe{k}.
\item \coqe{Mzero} is the empty set.
\item \coqe{Mstr} is the cartesian product of two sets.
\item \coqe{Mguard P m} is defined as \coqe{Mzero}, if \coqe{P} is false and otherwise \coqe{m}.
\end{itemize}

We use the monad operations to define queries as follows:
\begin{coq}
Definition query {S T: Type}
  (P : M S) (C : S -> bool) (E : S -> T) : M T :=
  Mbind P (fun x => Mguard (C x) (Mret (E x))).
\end{coq}
Here, \coqe{P} represents the {\tt for}-clause, \coqe{C} represents the {\tt where}-clause, and \coqe{E} represents the {\tt return}-clause.  As remarked earlier, our Coq library normalizes arbitrary combinations of the monad operations into the above form.

Embedded dependencies are defined in terms of queries by using the $front = back$ definition from the first half of the paper: 
\begin{coq}
Definition embedded_dependency {S S': Type}
  (F : M S) (Gf : S -> bool) (B : M S') (Gb : S -> S' -> bool)
:= Meq (query F Gf (fun x => x))
       (query (Mstr F B)
              (fun ab => Gf (fst ab) && Gb (fst ab) (snd ab))
              (fun x => fst x)).
\end{coq}



%We use the techniques described in later sections to normalize raw Coq code into this particular form.

\subsection{Background: Tactic-based Programming}
\label{sec:tactic-based}

The core of the Coq Proof Assistant is a pure and dependently-typed functional programming language, which by the Curry-Howard correspondence is also a logic.  In addition, Coq also features a partial, untyped ``tactic'' language called \ltac{} which can be used to construct Coq terms in a semi-imperative style.  \ltac{} is a meta-language for Coq, in much the same way that macros are a meta-language in C++.  The Coq terms that \ltac{} produces are checked by the Coq kernel in the same way that all Coq terms are checked by the kernel.  

In this paper, we propose to use \ltac{} to optimize queries.  This use of \ltac{} is somewhat unorthodox; the standard use of \ltac{} is to prove theorems via ``proof scripts'' that express how proofs should be built.  Indeed, the operational semantics of \ltac{} are tailored to fit this standard use of \ltac{}, rather than to fit our use of \ltac{} as a query optimizer.  For example, every \ltac{} tactic runs against a particular ``proof obligation'' (a.k.a goal), such as \coqe{x + y = y + x}.  In general, these goals are fully determined and it is simply the tactic's job to find a suitable proof.  In our case, instead of being given a fully specified goal, we will be given a partially specified goal that specifies a constraint on a ``unification variable'' that the tactic will seek to fill in while solving the proof.  It is the discovered values of these unification variables that will be our optimized queries.  We will demonstrate our particular way of using \ltac{} on a simple goal, where we want to find an optimized query for \coqe{?n}:
\begin{coq}
Meq ?n (x <- Movie ; y <- Movie ; ret x)
\end{coq}
Here, \coqe{?n} is the unification variable that we are trying to fill in and the constraint is that it must be equivalent to the query \coqe{x <- Movie ; y <- Movie ; ret x}.
We could simply pick \coqe{?n} by reflexivity to be the full query \coqe{x <- Movie ; y <- Movie ; ret x}, but that would require us to do an unnecessary join.  Instead, we will optimize the query (\coqe{?n}) by appealing to various query transformations that are provably correct.   For example, the following is a lemma that expresses that queries (in the set monad) are idempotent:
\begin{coq}
Lemma Mbind_dedup : forall {T U} (m : M T) (k : T -> M U),
  Meq (Mbind m k) (Mbind m (fun x => Mbind m (fun _ => k x))).
Proof. ... Qed.
\end{coq}
Because \coqe{Meq} is an equivalence relation, we can use \ltac's \coqe{setoid} \coqe{rewrite} to rewrite using \coqe{Mbind_dedup}.
Running \coqe{setoid_rewrite <- Mbind_dedup} on the goal above results in the new goal:
\begin{coq}
Meq ?n (x <- Movie ; ret x)
\end{coq}
Note that this does \emph{not} solve the unification variable \coqe{?n}; only the right-hand-side of the \coqe{Meq} is changed.
However, now we can pick \coqe{?n} to be \coqe{x <- Movie ; ret x} by reflexivity, and \coqe{?n} now does not require a join -- it has been optimized.  

There are often many equivalent ways to use \ltac{} to solve a particular goal.  In the example above, rather than first use \coqe{setoid_rewrite}, and then chose \coqe{?n} by reflexivity, we could also use Coq's \coqe{apply} tactic with \coqe{Mbind_dedup} to yield a proof in a single step.  We have found that when building our optimization procedure in \ltac{}, we typically use \coqe{apply} (or \coqe{eapply}) as the core reasoning tool, and we structure query optimization by applying theorems that express subproblems as premises and use \ltac{} ``chaining'' to solve these subproblems.  Consider the following (slightly contrived) example:


%Further, with the appropriate definitions, rewriting can happen deep within larger terms and can therefore be useful for operations such as re-associating binds or distributing operations.

%The ability for rewriting to apply deep within terms can also make it somewhat unpredicatable.
%It is up to Coq's unification heuristics to determine which term is rewritten if there are two possibilities.
%When we need to do more fine-tuned term manipulation we often prefer to use the more primitive \coqe{apply} tactic.
%Applying \coqe{Mbind_dedup} directly to this goal solves it with the same unification as above but this time instantiates \coqe{?n} with \coqe{x <- Movie ; ret x}.


\begin{coq}
Meq ?n (Mstr a b)
\end{coq}
If we wish to optimize \coqe{a} and \coqe{b} independently then we can \coqe{apply} the following lemma.
\begin{coq}
Lemma opt_plus : forall {T U} (m1 m1' : M T) (m2 m2' : M U),
  Meq m1' m1 ->
  Meq m2' m2 ->
  Meq (Mstr m1' m2') (Mstr m1 m2).
\end{coq}
When we \coqe{apply} this lemma to the above goal, we expect \coqe{m1} and \coqe{m2} to be completely determined but we expect \coqe{m1'} and \coqe{m2'} to be new unification variables.
%Because the tactic will introduce new unification variables, we use the \coqe{eapply} variant of the \coqe{apply} tactic.
Running \coqe{apply opt_plus} on the above goal yields the following sub-goals
\begin{center}
\begin{tabular}{ccc}
\lstinline!Meq ?m1' m1! & \qquad & \lstinline!Meq ?m2' m2! \\
\end{tabular}
\end{center}
while at the same time instantiating \coqe{?n} with \coqe{Mstr ?m1' ?m2'}.
As our tactics continue to fill in \coqe{?m1'} and \coqe{?m2'} (for example, during further optimization), \coqe{?n} will be automatically updated to reflect these changes.  


%% The ``standard'' approach to developing an algorithm such as the chase in Coq would be to program it in Gallina, Coq's pure functional programming language.
%% By the Curry-Howard correspondence~\cite{} this programming language is also a logic where propositions are types and programs are proofs.
%% A more complete description of the correspondence can be found in a variety of sources~\cite{}.
%% We will return to this approach in more detail in Section~\ref{sec:??}, in this work we develop our optimization in a different way.

%% In addition to Gallina, Coq also comes with another programming language, \ltac.
%% Unlike Gallina, which is strongly-typed and pure, \ltac\ is untyped and partial.
%% \ltac\ is a proof scripting language that is used to construct proof terms in an imperative style.
%% In this section we will discuss the key aspects of \ltac\ using a simple example for implementing addition.
%% \ltac\ is centered around manipulating Gallina terms.


%% The most common use of \ltac\ is to construct proofs of Gallina propositions.
%% For example, proving that \coqe{x + y = y + x} can be done by using the Gallina theorem that proves the commutativity of addition, i.e.
%% \begin{coq}
%% Theorem plus_comm : forall (n m : nat), n + m = m + n.
%% Proof. ... Qed.
%% \end{coq}
%% In \ltac{} we can apply this theorem using the \coqe{apply} tactic.
%% \begin{coq}
%% (* x, y : nat
%%  * ==============
%%  * x + y = y + x
%%  *)
%% apply plus_comm. (* => goal solved *)
%% \end{coq}
%% The implementation of \coqe{apply} unifies \coqe{x + y = y + x} with \coqe{?n + ?m = ?m + ?n} where the question mark variables (e.g. \coqe{?n}) are flexible unification variables that the unification can pick values for.
%% To solve this problem, the unification algorithm picks \coqe{?n $\mapsto$ x} and \coqe{?m $\mapsto$ y}.

%% In addition to \coqe{apply}, \ltac{} also provides \coqe{rewrite} to perform rewriting by user-defined relations.
%% Rewriting can be considerably more flexible than direct function application but works in much the same way.
%% Generalizing the goal above to \coqe{(x + y) * 3 = (y + x) * 3} makes \coqe{plus_comm} no longer immediately apply, but since equality is a transitive relation we can use \ltac{} to rewrite in the conclusion and then solve the goal by the reflexivity of equality.
%% \begin{coq}
%% (* x, y : nat
%%  * ==============
%%  * (x + y) * 3 = (y + x) * 3
%%  *)
%% rewrite plus_comm.
%% (* x, y : nat
%%  * ==============
%%  * (y + x) * 3 = (y + x) * 3
%%  *)
%% reflexivity. (* => goal solved *)
%% \end{coq}

%% In addition to solving concrete goals such as the ones above, \ltac{} is also able to manipulate unification variables directly.
%% Take the goal \coqe{?x = 3} for example.
%% The \coqe{reflexivity} tactic will solve this goal by instantiating \coqe{?x} with the value 3.
%% These unification variables are commonly used when proving existential quantifiers.
%% For example, in the following goal we use \coqe{eexists} to introduce a new unification varible for \coqe{x} and then solve the resulting equation with \coqe{reflexivity}.
%% \begin{coq}
%% (* ==================
%%  * exists x : nat , x = 3
%%  *)
%% $\texttt{\textcolor{dkblue}{eexists}}$.
%% (* ==================
%%  * ?x = 3
%%  *)
%% reflexivity. (* => goal solved *)
%% \end{coq}

%% \greg{this is a very unfocused section}

\begin{comment}
\subsection{Normalization}
\label{sec:normalization}

The first step of our optimization pipeline normalizes queries into the {\tt for}...{\tt where}...{\tt return} structure presented in Section~\ref{sec:queries}.
In Coq, we can define this structure as a function that takes the three pieces of the query and stitches them together into an instance.
The definition is the following:
\begin{coq}
Definition query {S T: Type}
  (P : M S) (C : S -> bool) (E : S -> T) : M T :=
  Mbind P (fun x => Mguard (C x) (Mret (E x))).
\end{coq}
Here, \coqe{P} represents the {\tt for}-clause that generates the tableau,
\coqe{C} represents the conditional {\tt where}-clause, and \coqe{E} represents the {\tt return}-clause.
In our representation, we use \coqe{Mstr} to construct a relational cross-product of two ``binds.''
Thus, the normalized form of our movies query is the following:
\begin{coq}
query (Mstr Movies Movies)
      (fun x => (fst x).(title) ?[=] (snd x).(title))
      (fun x => ((fst x).(director), (snd x).(actor)))
\end{coq}

%% In addition to generating this term, in order to fit into our fully-verified pipeline we must also generate a proof that guarantees that the transformation yields an equivalent instance.
%% Within the shallow encoding we can perform this normalization by incrementally applying proven theorems that witness the soundness of individual transformations.

%% The process begins by asking Coq to generate a unification variable representing the final answer and using it to witness the answer.
The phase starts on the goal \coqe{Meq ?1 q}.
The first theorem that we apply is an administrative theorem to massage the goal into the form expected by the rest of the normalization process.
%%  which we solve using a set of theorems crafted to specifically match on goals of this form.
\begin{coq}
Theorem prep_for_normal : forall {T} (q q' : M T),
  Meq q' (Mbind (query (Mret tt) (fun _ => true) (fun x => x))
                       (fun _ => q)) ->
  Meq q' q.
Proof. ... Qed.
\end{coq}
When applied to the initial goal, the first theorem (\coqe{prep_for_normal}), produces a new goal where left-hand side is unchanged and the right-hand-side of the equivalence now binds the empty query and then returns \coqe{q}.

While not particularly insightful from a proof-theoretic point of view, this transformation lays the groundwork for the remaining theorems to move components of \coqe{q} into the query.
For example, \coqe{normal_pull_plus} moves a bind from the instance and inserts it into the query.
\begin{coq}
Lemma normal_pull_plus
: forall {T U V W : Type} (qb : M T) (qg : T -> bool) (qr : T -> U) x (y : _ -> _ -> M V),
  Meq q'
      (Mbind (query (Mstr qb x) (fun x => qg (fst x)) (fun x => (qr (fst x), snd x)))
             (fun val : U * W => y (fst val) (snd val))).
  Meq q'
      (Mbind (query qb qg qr)
             (fun val : U => Mbind x (y val))).
Proof. ... Qed.
\end{coq}
Concretely, applying \coqe{normal_pull_plus} to the goal produced by \coqe{prep_for_normal} produces a new goal that has shrunk the size of the query to normalize (by ``removing'' a bind) and placing it in the {\tt for}-clause of the \coqe{query} definition.
\begin{coq}
Meq ?m (x <- query (Mstr (Mret tt) Movie) (fun _ => true) (fun x => x) ;
        y <- Movie ; guard ((snd x).(title) ?[=] y.(title)) ;
        ret ((snd x).(title), y.(actor)))
\end{coq}

It is important to notice that applying this theorem will happily lift non-atomic binds into the {\tt for}-clause.
For example, if applied directly to the goal \coqe{x <- (y <- Movies ; guard (y ?[=] 1) ; return y) ; return x}, \coqe{normal_pull_plus} will construct the following term that is not in the desired normal form.
\begin{coq}
x <- query (Mstr (Mret tt) (y <- Movies ; guard (y ?[=] 1)))
           (fun _ => true) Mret ;
return x
\end{coq}
To avoid this, we first perform a pre-processing phase that rewrites using a collection of lemmas which flatten nested binds and pushes guard expressions downwads toward the return.

Beyond lifting binds, the normalization process also includes lemmas for lifting {\tt where}- and {\tt return}-clauses into the query.
The later of these is slightly different because lifting the final {\tt return}-clause into the query marks the end of normalization.
Therefore, unlike the other lemmas, \coqe{normal_pull_ret} does not have a premise.
\begin{coq}
Lemma normal_pull_ret
: forall {T U V : Type} (qb : M T) qg (qr : T -> U) (y : _ -> V),
  Meq (Mbind (query qb qg qr)
             (fun val : U => Mret (y val)))
      (query qb qg (fun x => y (qr x))).
Proof. ... Qed.
\end{coq}


%% To handle nested structures such as  we preface the normalization step by rewriting with lemmas that flatten the query and push binds downward.
%% For example \coqe{Mbind_assoc} (shown below) converts the nested query above into \coqe{y <- Movies ; x <- return y ; return x}.
%% \begin{coq}
%% Lemma Mbind_assoc
%% : forall (A B C : Type) (c1 : M A) (c2 : A -> M B)
%%          (c3 : B -> M C),
%%   Meq (Mbind (Mbind c1 c2) c3)
%%       (Mbind c1 (fun x : A => Mbind (c2 x) c3))
%% \end{coq}

The final result of normalization for the movies query is exactly the query presented in Section~\ref{sec:example}.
In our syntax:
\begin{coq}
query (Mstr Movies Movies)
      (fun x => (fst x).(title) ?[=] (snd x).(title))
      (fun x => ((fst x).(director), (snd x).(actor)))
\end{coq}
\end{comment}


\subsection{The Chase as a Tactic}
\label{sec:ltac-chase}

Tableaux minimization makes heavy use of the chase, which we have implemented as a Coq tactic.  In a language-integrated query setting, implementing the chase as a tactic (as opposed to a Coq term) has two critical advantages.  First, the chase may never terminate, but all Coq programs do terminate, so any Coq (but not \ltac{}) implementation of the chase must be explicitly bounded by some number of chase steps.  Second, and more importantly, {\it Coq tactics can appeal to other Coq tactics.}  As we saw in the indexing example in the first half of the paper, running the chase can involve reasoning over predicates such as $<$ that appear in where clauses.  If we were implementing the chase as a Coq program (such as in cite), we would be forced to choose some predetermined set of predicates and implement, as part of the Coq program implementing the chase, a reasoning procedure for terms involving these predicates.  By implementing the chase as a tactic, we can appeal to Coq's \coqe{omega} tactic for reasoning about $<$, for example.  Moreover, the particular predicates and associated tactics need not be determined in advance; users are free to write arbitary Coq code in where clauses, and to construct hint databases for how to reason about the terms in where clauses.  

Our chase tactic applies to goals that are contingent on EDs.  For example, we can optimize with the \coqe{title_director} ED using the following goal structure.
\begin{coq}
title_director ->
Meq ?q
    (m1 <- Movies ; m2 <- Movies ;
     guard (title m1 = title m2) ;
     return (director m1, actor m2))
\end{coq}

%% Chasing a query requires solving 4 problems in order:
%% \begin{enumerate}
%% \item the tactic must find an ED to chase with (Section~\ref{sec:traverse-ed}).
%% \item the tactic must find a homomorphism from the front of the ED to the binders of the query (Section~\ref{sec:morphism-find}).
%% \item the tactic must check that the {\tt where}-clause of the ED implies the {\tt where}-clause of the query under the mapping that our tactic found in the previous step (Section~\ref{sec:side-condition}).
%% \item the tactic must ensure that chasing will introduce new information (Section~\ref{sec:progress}); i.e., the tactic must terminate when a fixed point is reached.
%% \end{enumerate}

%% We will now examine each of these steps in turn.

\subsubsection{Finding an Embedded Dependency}
\label{sec:traverse-ed}

In the movie example, there is only one embedded dependency to chase with, but in more complex examples such as the indexing example there are multiple EDs.  To find an appropriate ED, we can simply exhaustively consider all of the user-supplied EDs.  However, demonstrating this exhaustive enumeration serves as a useful primer for the more complicated next step of finding a homomorphism.  This step also demonstrates just how counter-intuitive it can be to program even a small part of a query optimizer using \ltac{}!

When there are multiple candidate EDs \coqe{A}, \coqe{B}, and \coqe{C}, and we wish to optimize the query \coqe{q}, the goal posed to the tactic is the following:
\begin{coq}
A /\ B /\ C -> Meq ?q q
\end{coq}

Two lemmas allow us to exhaustively search these EDs.
\begin{coq}
Lemma ed_pick_left : forall {A B C : Prop},
  (A -> C) ->
  A /\ B -> C.
Proof. ... Qed.
Lemma ed_pick_right : forall {A B C : Prop},
  (B -> C) ->
  A /\ B -> C.
Proof. ... Qed.
\end{coq}
\coqe{ed_pick_left} focuses on the left-hand-side of a conjunction while \coqe{ed_pick_right} focuses on the right-hand-side.
When only a single ED remains, we can start the actual chase procedure.
We express the backtracking search using \ltac's \coqe{+} combinator.
The following tactic performs the search.
\begin{coq}
Ltac ed_search :=
  lazymatch goal with
  | $\vdash$ _ /\ _ -> _ =>
    (  simple apply ed_pick_left
     + simple eapply ed_pick_right) ; ed_search
  | $\vdash$ _ -> _ => idtac
  end.
\end{coq}
Here, the \coqe{lazymatch goal with} is performing syntactic matching of the goal against the candidate patterns selecting the first that matches.
The first branch chooses either the left- or right-side using the above tactics and then continues the search.
The second branch is the default case which finishes the search when the premise does not contain a conjunction.

%% The previous sections havee shown how to chase a single embedded dependency, chasing a collection of EDs is not much more difficult.
%% Embedded dependencies are communicated via premises on the entailment, so all that remains is to non-deterministically select the appropriate ED.
%% The approach here is the same as when we non-deterministically found the morphism; we simply perform an exhaustive backtracking search trying to chase each one until we succeed.
%% For convenience we require that multiple EDs are combined with $\wedge$.

\subsubsection{Running the Chase Step}
\label{sec:chase-step}

Once we have isolated a single ED to attempt to chase, we need to apply a prepping lemma to get the goal into the appropriate form.
For the chase step, this prepping step involves applying the following theorem which expresses the soundness of the chase.
\begin{coq}
Theorem chase_sound {S S' T U}
  (P : M S) (C : S -> bool) (E : S -> T)
  (F : M S') (Gf : S' -> bool) (B : M U) (Gb : S' -> U -> bool)
: forall (h : S -> S'),
    Mimpl (Mmap h P) F ->
    (forall x, C x = true -> Gf (h x) = true) ->
    embedded_dependency F Gf B Gb ->
    Meq (query P C E)
        (query (Mstr P B)
               (fun ab : S $\times$ U => C (fst ab) &&
                                  Gb (h (fst ab)) (snd ab))
               (fun ab => E (fst ab))).
\end{coq}
In this theorem, all of the variables in the first three lines are given explicitly once we know the ED and the query that we are chasing.  The next three lines (starting with the colon) express the tableau homomorphism from the front of the ED to the query.
In the next two paragraphs we explain how we compute this homomorphism.

\paragraph{Finding a Homomorphism}
In the definition of a chase step in Section~\ref{sec:chase} a homomorphism is a map from variables bound in the front of the embedded dependency to the variables bound in the {\tt for}-clause of the query.  In our movie query example this corresponds to a mapping that assigns values in $\{x,y\}$ to values in $\{m_1,m_2\}$.

Since our queries are simply Coq terms, referencing these binders explicitly can be quite difficult, and is not very extensible.
Rather, we will encode our mapping of binders as a function between the types being bound.
In this example, the type of the {\tt for}-clause of the query is \coqe{Movie * Movie} and the type of the {\tt forall}-clause is also \coqe{Movie * Movie}, so we are looking for a function \coqe{h : Movie * Movie -> Movie * Movie}.
Looking at the query and the ED, there are 4 reasonable choices.
\begin{coq}
h x = (fst x, fst x)
h x = (fst x, snd x)
h x = (snd x, fst x)
h x = (snd x, snd x)
\end{coq}
%Where \coqe{fst} extracts the first element of the pair and \coqe{snd} extracts the second.

As is customary for our specialized use of \ltac, we are going to construct these functions incrementally using theorems that represent individual steps of reasoning.
The search is much the same as the search for isolating a particular ED.
There are two main differences.
The first is that we must now find a binder for each binder in the front of the ED.
The second is that in doing so, we must explicitly construct a substitution \coqe{h} to use to check that \coqe{h} is actually a homomorphism.
When searching for the homomorphism, the goal will have the following form, where \coqe{P} is the {\tt for}-clause of the query and \coqe{F} is the {\tt forall}-clause in the ED.
\begin{coq}
Mimpl (Mmap ?h P) F
\end{coq}
In the above \coqe{Mimpl}, is defined as subset containment (or rather, the analog of subset containment for arbitrary monads), and \coqe{Mmap} expresses the application of a substitution.  The first task is to break \coqe{F} down into atomic units which, essentially, correspond to the binders.
The \coqe{pick_split} lemma applies when \coqe{F} is formed from a \coqe{Mstr}.
\begin{coq}
Lemma pick_split
: forall {T U U' : Type} (m : M T) (u : M U) (u' : M U') f g,
  Mimpl (Mmap f m) u ->
  Mimpl (Mmap g m) u' ->
  Mimpl (Mmap (fun x => (f x, g x)) m) (Mstr u u').
Proof. ... Qed.
\end{coq}
This lemma states that we can find a morphism from \coqe{m} to \coqe{Mstr u u'} if we can find a morphism from \coqe{m} to \coqe{u} and from \coqe{m} to \coqe{u'}.
Note that in addition to breaking down the morphism by decomposing it into \coqe{f} and \coqe{g}, the left-hand-side of the implication also shows how to construct the final homomorphism given \coqe{f} and \coqe{g}.

Reapeatedly applying \coqe{pick_split} will eventually break the {\tt forall}-clause down into atomic elements that we can match up with the query.
This matching is essentially the same as the ED search procedure except that, as above, we must record the way to reconstuct the \coqe{h} function.
\begin{coq}
Lemma pick_left
: forall {T' U' V} (f' : U' -> V) (x : M V) (y : M T') (k' : M U'),
  Mimpl (Mmap f' k') x ->
  Mimpl (Mmap (fun x => f' (fst x)) (Mstr k' y)) x.
Proof. ... Qed.

Lemma pick_right
: forall {T' U' V} (f' : U' -> V) (x : M V) (y : M T') (k' : M U'),
  Mimpl (Mmap f' k') x ->
  Mimpl (Mmap (fun x => f' (snd x)) (Mstr y k')) x.
Proof. ... Qed.

Lemma pick_here
: forall {T} (x : M T), Mimpl (Mmap (fun x => x) x) x.
Proof. ... Qed.
\end{coq}
\coqe{pick_left} decides to use only the left-hand side of the \coqe{Mstr k' y} to determine \coqe{x}, \coqe{pick_right} is analagous for the right-hand side.
Finally, \coqe{pick_here} applies when the value being searched for is exactly the value being bound in which case it can pick the value directly.

%% We combine these proofs into into a search using \ltac's backtracking \coqe{+} operator.
%% The core of the procedure is the following:
%% \begin{coq}
%% Ltac find_bind_morphism :=
%%   lazymatch goal with
%%   | |- Mimpl (Mmap _ _) (Mstr _ _) =>
%%       (eapply pick_split)
%%   | |- Mimpl _ _  =>
%%       (simple eapply pick_here)
%%     + (simple eapply pick_left)
%%     + (simple eapply pick_right)
%%   end ; find_bind_morphism.
%% \end{coq}
%% Applied to our simple example this essentially amounts to:
%% \begin{coq}
%% eapply pick_split ;
%%   (simple eapply pick_here + simple eapply pick_left + simple eapply pick_right) ;
%%   (simple eapply pick_here + simple eapply pick_left + simple eapply pick_right)
%% \end{coq}
%% The first application splits the morphism into two parts, and the semi-colon runs the remainder of the tactic on the two independent goals.
%% The plus operator effectively allows choices to be made independently and allows backtracking to encode the non-deterministic choice.
%% In this case, along the first goal we \coqe{pick_left} and then \coqe{pick_here} and along the second we \coqe{pick_right} and then \coqe{pick_here}.

The search completes with each of the four candidate substitutions written above, the next step is to attempt to solve the side condition (i.e., that a candidate is in fact a homomorphism).

\paragraph{Solving the Side-conditions}
With a candidate substitution in hand, the next step is to discharge the side condition which guarantees that the {\tt where}-clause of the embedded dependency implies the {\tt where}-clause of the query.  In our movies example, this amounts to the following (we will write, for example, \coqe{z.title} instead of \coqe{title z} in an effort to make things more readable):
\begin{coq}
forall x : Movie $\times$ Movie, (fst x).title = (snd x).title = true
     -> (fst (h x)).title = (snd (h x)).title
\end{coq}

However, once we get to this step, \coqe{h} is exactly one of the morphisms constructed by the previous step.
When we plug in the correct morphism, i.e. \coqe{h x = (fst x, snd x)} (recall that we are enumerating all of the potential morphisms) we get the following:
\begin{coq}
forall x : Movie $\times$ Movie, (fst x).title = (snd x).title = true
     -> (fst x).title = (snd x).title = true
\end{coq}

While this goal is trivially true, in general these side conditions can require potentially arbitrary reasoning.
In order to make the automation extensible, the chase tactic is parameterized by the tactic to use to discharge this side-condition.
For example, in the indexing example from Section~\ref{sec:minimize} we must prove the following implication which relies on arithmetic reasoning.
\begin{coq}
forall p, p.age ?[>] 16 && p.age ?[<] 18 = true ->
          p.age ?[<] 21 = true
\end{coq}
Coq already has the \coqe{omega} tactic for performing arithmetic reasoning.
In addition, we can develop our own theorems and \ltac{} for automating different domains, for example the length of strings, the case of characters, or even complex arithmetic on floating point numbers.

\paragraph{Ensuring Progress}
After the side-condition is checked, the final piece is to ensure that this chase step makes progress by adding \emph{new} information or data to the query.
Semantically, we express this by ensuring that the new query is not isomorphic to the old query.
Checking this isomorphism \emph{between queries} is essentially the same as what we have done up to this point except that we must also show that the morphism preserves the {\tt return}-clause of the query under the assumptions in the {\tt where}-clause.
%% This check simply requires that we have a tactic that can find an isomorphism between queries, which is essentially the same as what we have done up until this point, except that we also need to prove that the \RETURN-clauses of the two queries are equal under the assumptions in the \WHERE-clause.
The \ltac{} to check this is trivial given the machinery that we developed in the previous two steps.
The entire \ltac{} is (essentially) the following:
\begin{coq}
Ltac prove_query_morphism solver :=
  eapply check_query_morphism_apply ;
    [ find_bind_morphism
    | simpl ; solve [ solver ]
    | simpl ; solve [ solver ] ]).

Ltac prove_query_isomorphism solver :=
  split; prove_query_morphism solver.
\end{coq}
Note again that these tactics are parameterized by the underlying solver (\coqe{solver}) that that will use to discharge the side-conditions.

\subsubsection{Computing the Fixedpoint}
With the ability iterate over the EDs (Section~\ref{sec:traverse-ed}) and compute a chase step (Section~\ref{sec:chase-step}), it is simple to implement the entire chase.
The key detail to be aware of when constructing the full chase lies in delimiting the backtracking that the algorithm requires.
When we fail to chase with a particular ED, it is no longer necessary to backtrack into that search until we have added new information.
To ensure that we do not backtrack between two different chase steps, we use \ltac's \coqe{once} and \coqe{first} tacticals.
\begin{coq}
repeat first
   [ eapply transitive_refine_conditional ;
     [ solve [ once (ed_search ; chase_step solver) ]
     | ]
   | eapply reflexive_refine_conditional ].
\end{coq}
Here, \coqe{repeat} repeats this until the goal is solved or the tactic fails.
\coqe{first [ a | b ]} runs \coqe{a} and, if it fails, runs \coqe{b}.
The \coqe{transitive_refine_conditional} theorem applies transitivity speculating progress while leaving the door open to further optimization.
The first of the two goals is solved using the tactic on the second line.
Here, \coqe{solve} requires that the inner tactic completely solves the goal, and \coqe{once} prevents backtracking triggered by later pieces of the search.
If progress can not be made in the chase, then the goal is solved by the second branch of the \coqe{first} tactical which picks the current query as the result.

%Note that by developing this procedure in \ltac{} there is no way to reason about whether it terminates.

\subsection{Minimization}
\label{sec:minimization}

The final step in optimization is to remove extraneous binds from the query.
By now, most of the techniques have already been discussed.
We will use incremental lemmas to iterate through the binders and attempt to drop each one by expressing a side-condition the expresses that the information in that binder can be reconstructed from the other bound values.
The core relevant lemmas are the following:
\begin{coq}
Lemma minimize_drop
: forall {T T' V : Type} (qb : M T) (qb' : M T') qg (qr : _ -> V) f (qb'' : M T') qg'',
   Find f
-> Meq (query (Mstr qb qb') qg qr)
       (query qb' (fun y => qg (f y,y)) (fun y => qr (f y,y)))
-> Meq (query qb' (fun y => qg (f y,y)) (fun y => qr (f y,y)))
       (query qb'' (fun y => qg'' (f y,y)) (fun y => qr (f y,y)))
-> Meq (query (Mstr qb qb') qg qr)
       (query (Mmap (fun y => (f y, y)) qb'') qg'' qr).
Proof. ... Qed.
\end{coq}
\coqe{minimize_drop} states that we can drop the first (left) binder if we can find a way to compute it (\coqe{f}) from the right binder, i.e. a morphism from \coqe{Mmap f qb'} to \coqe{qb}.
The first premise of this theorem, \coqe{Find f}, is a dummy premise; it is trivially true.
We include it in order to simplify constructing it using tactics.
For example, we can easily write lemmas that parallel \coqe{pick_left}, \coqe{pick_right}, and \coqe{pick_here}.
The second premise, ensures that this choice of \coqe{f} respects the equivalence of the query.
To solve it, we ``back chase'' the left-hand side of the equivalence and try to determine if it is homomorphically equivalent to the right-hand side.

\coqe{minimize_keep} is the fallback case.
\begin{coq}
Lemma minimize_keep
: forall {T T' V : Type} (qb : M T) (qb' : M T') qg (qr : _ -> V) (qb'' : M T') qg'',
  (forall x : T,
   Meq (query qb' (fun y => qg (x,y)) (fun y => qr (x,y)))
       (query qb'' (fun y => qg'' (x,y)) (fun y => qr (x,y)))) ->
  Meq (query (Mstr qb qb') qg qr)
      (query (Mstr qb qb'') qg'' qr).
Proof. ... Qed.
\end{coq}
If we can not find a way to construct the information from the rest of the query, then we must keep this information, but we can still optimize the rest of the query (represented by quantifying over any value from the relation and ensuring that the rest of the query is still valid).


%\subsection{Extension}
%
%\greg{This is on the nested relational queries}
%
%One of the main benefits to implementing our query optimizer in \ltac{} is the ability to easily extend the optimizer simply by proving relatively simple lemmas such as the ones in the previous section.
%For example, the above \coqe{normal_pull_plus} is not sufficient to handle nested queries where \coqe{x} could depend on the values that were previously bound.
%Normally, supporting this type of term manipulation would be quite painful since we would need to track the number of binders that we are under or generate fresh names.
%Instead, we can foist all of that complexity back onto Coq by phrasing the appropriate lemma and using Coq's higher-order unification to solve the problem for us.
%Using this technique, the generalization of \coqe{normal_pull_plus} to handle nested relations would be the following:
%\begin{coq}
%Lemma normal_pull_dplus_ret_id
%: forall {T U V W : Type} (qb : M T) qg x (y : _ -> _ -> M V),
%  Meq (Mbind (query qb qg (fun x => x))
%             (fun val : T => Mbind (x val) (y val)))
%      (Mbind (query (Mdplus qb x) (fun x => qg (fst x)) (fun x => (fst x, snd x)))
%             (fun val : T * W => y (fst val) (snd val))).
%Proof. ... Qed.
%\end{coq}


%% \subsection{Non-Semantic Optimization}
%% \greg{not finished but could be interesting and relatively simple}

\section{Discussion}
\label{sec:discussion}

The approach that we take in this work is to use tactic-based programming to develop a verifying implementation of semantic query optimization.
Tactic-based programming has several trade-offs when compared to more traditional development and verification techniques.

The primary benefit of tactic-based programming is the ability to easily extend algorithms with a minimal amount of work.
Much of this benefit is due to the flexibilty of working indirectly on Coq's underlying terms.
For example, we can extend the chase algorithm with support for nested relations simply by proving new lemmas that show how to locally manipulate terms.
A more traditional implementation would have to adjust the term representations to support the more sophisticated structure of queries and then update all of the algorithms to work on the new representation.
Along another vein, we are able to support arbitrary Coq computation within guards.
While the examples are restricted to using equality and less-than, there is nothing preventing us from reasoning about more complex operations such as the case of characters or the length of strings.

Another benefit of tactic-based programming is that we get to re-use much of the underlying Coq infrastructure.
In particular, features such as higher-order unification, existing automation, and \ltac's backtracking search mechanism are all useful when building optimization procedures.
We have found, in particular, that the new backtracking proof search facilities, namely \coqe{+} and dependent goals, introduced in Coq 8.5 are particularly useful for this type of development.
Without this feature, tactics can not backtrack between different goals which initially forced us to more meticulously maintain goals in addition to forcing us to code in continuation passing style, rather than a more direct style.

There are also drawbacks to tactic-based development.
First, tactics are completely untyped which makes it cumbersome to track down errors which are often due to typos.
While simple types would help track some information, thoughout the course of development we found that one of the most cumbersome parts of development is keeping track of what the goal looks like.
For example, whether the unfication is on the left or the right of the \coqe{Meq} influences the way that you write lemmas.
Similarly, many lemmas had to be duplicated to handle extra bits of context.
For example, we had to write separate lemmas for chaining together refinements in the presence and absence of embedded dependencies.
Several authors have proposed more richly typed tactic-based programming languages, noteably Mtac~\cite{ziliani2013mtac} and VeriML~\cite{stampoulis2010veriml}.
While neither of these are as mature or rich as \ltac, it would be interesting to explore whether their features greatly the development process.

Another problem inherent in tactic-based programming is speed.
Figure~\ref{fig:performance} shows the time it takes to optimize the queries presented in this paper.
The {\tt normalize} task (not discussed) is the time it takes to convert raw Coq expressions into the form \coqe{query P C R}.
The {\tt chase} and {\tt minify} phase are the phases described in Sections~\ref{sec:chase-step} and~\ref{sec:minimization}.
The final {\tt simplify} phase performs rudimentary non-semantic optimization, for example, converting the query
\begin{coq}
query (Mmap (fun x => (x,x)) Movie) (fun _ => true)
      (fun x => fst x)
\end{coq}
into
\begin{coq}
query Movie (fun _ => true) (fun x => x)
\end{coq}
.
The {\tt Total} row is the total of all of the phases.

\begin{figure}
\centering
\begin{tabular}{l | r | r}
               & \multicolumn{2}{| c}{Time (s)} \\
\textbf{Phase} & \textbf{Movie} & \textbf{Index} \\\hline
Normalize      & 0.71  & 0.55 \\
Chase          & 1.17  & 5.16 \\
Minimize       & 1.53  & 45.60 \\
Simplify       & 0.11  & 0.27 \\\hline
\textbf{Total} & 4.51  & 51.90 \\
\end{tabular}

\caption{Performance of the optimizer.}
\label{fig:performance}
\end{figure}

Figure~\ref{fig:performance} shows the time it takes to optimize the two queries.
The {\tt minimize} phase is the longest which is the result of needing to perform the back-chase before testing homomorphism equality.

\paragraph{A Verified Implementation}
One of the benefits of working in a rich logic such as Coq is the ability to leverage dependent types to offset some of the burden of building proofs.
For example, rather than writing tactics to compute the chase and then use Coq's logic to prove the procedure sound similar to the work of Benzaken~\cite{coqdb}.
This can yield a substantial improvement in performance.
For example, optimizing the Movie query from beginning to end is practically instantaneous using Coq's \coqe{vm_compute} reduction mechanism~\cite{gregoire2002vmcompute}.

The price we pay for this speed is the need to reimplement some of Coq's internal features such as unification.
For example, while we were able to quickly write a basic tautology solver capable of discharging the side conditions needed for the movies example, it would havce been quite a bit more work to implement a procedure capable of reasoning about numbers necessary for the indexing example.
Fundamentally, however, there is nothing that prevents us from performing this kind of reasoning within the logic, and a variety of work~\cite{malecha2015thesis,besson2007micromega,braibant2011aac,lescuyer2009sat,lescuyer2011these} has shown it to be both an efficient and powerful mechanism.


%% There has been some work in marrying the beneficial parts of tactic-based and standard programming within a proof assistant.
%% \textsc{MirrorCore} defines a deep embedding of a simply typed language within Coq and builds a tactic language, \textsc{Rtac}, on top of it~\cite{malecha2015thesis}.
%% Emperical results show that this can yield orders of magnitude performance improvement on some problems and provides more flexibility to directly manipulate terms than the tactic-base approach that we use in this work.


\section{Conclusions}

In this work we developed a verifying semantic query optimizer using the chase.
Our approach leverages programming with tactics to manipulate raw Coq terms and simultaneously produce an optimized query and a proof that the query has the same meaning as the input query.
Performing this optimization using tactics has its own set of problems.
For example, terms need to be manipulated indirectly through applying and rewriting, backtracking search is necessary to explore multiple potential optimization paths, and traditional first-order phrasings of syntax manipulation in terms of binders needs to be translated to manipulations of environments.
By solving these problems, we leave the optimzation open to extension through other tactics.
For example, our reasoning already makes use of Coq's \coqe{omega} tactic for reasoning about natural numbers and extending it with other smarts is not difficult.

While not currently competivitve with more traditional programming languages in terms of speed, the features built into \ltac{} makes it a good choice to implement optimizations of this variety.
Looking forward, tactic-based programming is likely to improve.
It already features prominently in the development of Idris~\cite{brady2013idris}, and recent work~\cite{malecha2015thesis,devriese2013tsmp,vanderwalt2013engineering-reflection-agda} has shown how tactic-based programming can be made first-class given sufficiently rich type systems.
As these systems mature, they will begin to develop their own customizable and extensible automation libraries that will enable them to be used in the same ways that we currently use scripting languages.
This next wave of tactic-based languages and libraries are already demonstrating substantial performance improvements and it is likely that it won't be too long before semantic optimization will begin to make it into mainstream systems.



%% Todo: need to hammer home why writing the chase as a tactic is hard, and why doing it that was is valuable.  This is a good place to compare to MirrorCore~\cite{malecha2014thesis}, rather than the related work.



{\bf Acknowledgement.}  The authors would like to thank Lucian Popa for answering many questions about semantic optimization.

\bibliographystyle{plain}
\bibliography{thesisbib}


\end{document}
